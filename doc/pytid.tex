\documentclass[a4paper,oneside]{book}
\newenvironment{abstract}{}{}
\usepackage{abstract}
\usepackage{noweb}
% Needed to relax penalty for breaking code chunks across pages, otherwise 
% there might be a lot of space following a code chunk.
\def\nwendcode{\endtrivlist \endgroup}
\let\nwdocspar=\smallbreak

\usepackage[hyphens]{url}
\usepackage{hyperref}
\usepackage{authblk}

%\usepackage[utf8]{inputenc}
%\usepackage[T1]{fontenc}
\usepackage[swedish,british]{babel}
\usepackage{booktabs}

\usepackage[natbib,style=verbose,citestyle=verbose,maxbibnames=99]{biblatex}
\addbibresource{bibliography.bib}

\usepackage[inline]{enumitem}
\setlist[enumerate]{label=(\arabic*)}

\usepackage[all]{foreign}
\renewcommand{\foreignfullfont}{}
\renewcommand{\foreignabbrfont}{}

%\usepackage{newclude}
\usepackage{import}

\usepackage[strict]{csquotes}
\usepackage[single]{acro}

\usepackage{subcaption}

\usepackage[noend]{algpseudocode}
\usepackage{xparse}

\let\email\texttt

\usepackage[outputdir=ltxobj]{minted}
\setminted{autogobble,linenos}

\usepackage{pythontex}
\setpythontexoutputdir{.}
\setpythontexworkingdir{..}

\usepackage{fancyvrb}

\usepackage{amsmath}
\usepackage{amssymb}
\usepackage{mathtools}
\usepackage{amsthm}
\usepackage{thmtools}
\usepackage[unq]{unique}
\DeclareMathOperator{\powerset}{\mathcal{P}}

\usepackage[binary-units]{siunitx}

\usepackage{pgf}

\usepackage[capitalize]{cleveref}
\crefalias{question}{item}
\crefname{question}{question}{questions}
\Crefname{question}{Question}{Questions}
\creflabelformat{question}{#2\textup{#1}#3}

% adapted from https://tex.stackexchange.com/a/30726/17418
\newcommand\chnote[1]{%
  \begingroup
  \renewcommand\thefootnote{}\footnote{#1}%
  \addtocounter{footnote}{-1}%
  \endgroup
}

\usepackage[marginparmargin=right]{didactic}


\title{%
  A schedule management tool for teaching
}
\author{%
  Daniel Bosk
}
\affil{%
  KTH EECS\\
  \texttt{dbosk@kth.se}
}

\begin{document}
\frontmatter
\maketitle

\vspace*{\fill}
\VerbatimInput{../LICENSE}
\clearpage

\begin{abstract}
  We provide a library and command-line interface to manage TA bookings for 
teaching sessions.

The library can parse ICS calendars containing the course schedule, generate a 
sign-up sheet for TAs, later parse that sign-up sheet (directly from a Google 
Sheets URL) to compute employment data for amanuensis or time sheets for hourly 
employed TAs, among other things.


\end{abstract}
\clearpage

\tableofcontents
\clearpage

\mainmatter
\chapter{Introduction}

LADOK (abbreviation of \foreignlanguage{swedish}{Lokalt Adb–baserat 
DOKumentationssystem}, in Swedish) is the national documentation system for 
higher education in Sweden.
This is the documented source code of \texttt{ladok3}, a LADOK3 API wrapper for 
Python.

The \texttt{ladok3} library provides a non-GUI application that, similar to 
access via a web browser, only provides the user with access to the LADOK data 
and functionality that they would actually have based on their specific user 
permissions in LADOK.
It can be seem as a very streamlined web browser just for LADOK's web 
interface.
While the library exploits caching to reduce the load on the LADOK server, this 
represents a subset of the information that would otherwise be obtained via
LADOK's web GUI export functions.

You can install the \texttt{ladok3} package by running
\begin{minted}{bash}
pip3 install ladok3
\end{minted}
in the terminal.
You can find a quick reference by running
\begin{minted}[firstnumber=last]{bash}
pydoc ladok3
\end{minted}

We provide the main class \texttt{LadokSession} (\cref{LadokSession}).
The \texttt{LadokSession} class acts like \enquote{factories} and will return 
objects representing various LADOK data.
These data objects' classes inherit the \texttt{LadokData} (\cref{LadokData}) 
and \texttt{LadokRemoteData} (\cref{LadokRemoteData}) classes.
Data of the type \texttt{LadokData} is not expected to change, unlike 
\texttt{LadokRemoteData}, which is.
Objects of type \texttt{LadokRemoteData} know how to update themselves, \ie fetch 
and refresh their data from LADOK.
When relevant they can also write data to LADOK, \ie update entries such as 
results.

One design criteria is to improve performance.
We do this by caching all factory methods of any \texttt{LadokSession}.
The \texttt{LadokSession} itself is also designed to be cacheable: if the session to 
LADOK expires, it will automatically reauthenticate to establish a new session.



\part{The library}

\input{../src/pytid/schedules/init.tex}
\input{../src/pytid/schedules/utils.tex}


\part{A command-line interface}

\input{../src/pytid/cli/init.tex}



\backmatter
\printbibliography


\end{document}
